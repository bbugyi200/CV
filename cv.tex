%!TEX TS-program = xelatex
% vim: spell spelllang=en_us
\documentclass[]{friggeri-cv}
\addbibresource{bibliography.bib}

\usepackage{graphicx}
\usepackage{calc}
\usepackage{enumitem}

\setitemize{itemsep=3pt}

\graphicspath{{images/}}

\newlength\myheight
\newlength\mydepth
\settototalheight\myheight{Xygp}
\settodepth\mydepth{Xygp}
\setlength\fboxsep{0pt}
\newcommand*\inlinegraphics[2]{%
  \settototalheight\myheight{Xygp}%
  \settodepth\mydepth{Xygp}%
  \raisebox{-\mydepth}{\includegraphics[height=\myheight+#2pt]{#1}}%
}

\begin{document}
\header{bryan}{bugyi}{}

% In the aside, each new line forces a line break
\begin{aside}
  \section{contact}
    (609)500-7081
    \href{mailto:bryanbugyi34@gmail.com}{bryanbugyi34@gmail}
    \href{mailto:bryan.bugyi@rutgers.edu}{bryan.bugyi@rutgers}
  \section{web}
    \href{https://github.com/bbugyi200}{github://bbugyi200}
    \href{http://facebook.com/bryan.bugyi}{fb://bryan.bugyi}\vspace{0.3cm}
    \href{http://bryanbugyi.pythonanywhere.com}{bryanbugyi.\\pythonanywhere.com}
  \section{programming}
    \inlinegraphics{heart.jpg}{5} Python
    C/C++
    Java
    Haskell
    Bash
    MATLAB
    HTML \& CSS
 \section{tools}
	Linux
	\LaTeX
	Git
	MySQL
\end{aside}

\section{summary}
    { \small
        Student of Computer Science and Mathematics at Rutgers University, New Brunswick. Professionally proficient in Python. Generally proficient in many other languages. Dedicated to writing clean, efficient code using best software development practices.
    }

\section{projects}
\begin{entrylist}
    \entry
      {2017-2018}
      {SlackRU \inlinegraphics{python.png}{2}\inlinegraphics{sql.png}{0}}
      {\href{https://github.com/HackRU/SlackRU/tree/develop}{HackRU/SlackRU}}
      {Slackbot designed to respond to user questions and inquiries as well as to match users based on common skill sets. Expected to be used by hundreds of users at the upcoming HackRU in the Spring. Built using Slack's API to collaborate with a backend SQL database hosted on a web server powered by Python's Flask web framework.}
    \entry
      {2017-2018}
      {LocalAlias\, \inlinegraphics{bash.png}{1}}
      {\href{http://github.com/bbugyi200/LocalAlias}{bbugyi200/LocalAlias}}
      {\vspace{-0.25cm}}
    \entry
      {2017-2018}
      {ProtectMyLaptop \inlinegraphics{python.png}{2} \inlinegraphics{bash.png}{1}}
      {\href{http://github.com/bbugyi200/ProtectMyLaptop}{bbugyi200/ProtectMyLaptop}}
      {\vspace{-0.25cm}}
    \entry
      {2016-2017}
      {Wumpus World \inlinegraphics{python.png}{2}}
      {\href{http://github.com/bbugyi200/wumpusworld}{bbugyi200/WumpusWorld}}
      {An attempt at designing an agent that can logically navigate the Wumpus World environment described in \emph{Artificial Intelligence: A Modern Approach.}}
    \entry
      {2016-2017}
      {Personal Website \inlinegraphics{python.png}{2} \inlinegraphics{html.png}{0} \inlinegraphics{css.jpg}{0}}
      {\href{http://bryanbugyi.pythonanywhere.com}{bryanbugyi.pythonanywhere.com}}
      {Designed using HTML, CSS, and Python's Flask web framework.}
    \entry
      {2016}
      {IntelliBudget \inlinegraphics{python.png}{2}\inlinegraphics{sql.png}{0}}
      {\href{https://github.com/bbugyi200/IntelliBudget}{bbugyi200/IntelliBudget}}
      {A platform independent personal budgeting application complete with a graphical user interface (implemented using Python's Tkinter library) and an SQL database designed to store user expense data.}
\end{entrylist}

\section{education}
\begin{entrylist}
  \entry
	{2017-2019}
	{Bachelor of Science}
	{Rutgers University, New Brunswick}
	{Computer Science and Mathematics}
  \entry
    {2015-2017}
    {Associate of Science {\footnotesize (4.0 GPA)}}
    {Rowan College at Burlington County}
    {Computer Science}
\end{entrylist}

\section{experience}
\begin{entrylist}
  \entry
    {2011-2016}
    {Comcast Cable}
    {Mount Laurel, NJ}
    {Tier III Technical Support}
\end{entrylist}

\section{activities}
    \begin{entrylist}
        \entry
        {2017-2018}
        {HackRU Architect Team}
        {We Build Tools for HackRU (Rutgers Hackathon)}
        {\vspace{-0.3cm}}
        \entry
        {2017-2018}
        {USACS Mentorship Program}
        {Mentor for CS Underclassmen}
        {\vspace{-0.3cm}}
        \entry
        {2016-2017}
        {Undergraduate Research in Mathematics}
        {Advised by Professor Weisbrod}
        {}
    \end{entrylist}
    \vspace{-1.0cm}
\end{document}
