% (c) 2002 Matthew Boedicker <mboedick@mboedick.org> (original author) http://mboedick.org
% (c) 2003-2007 David J. Grant <davidgrant-at-gmail.com> http://www.davidgrant.ca
% (c) 2008 Nathaniel Johnston <nathaniel@nathanieljohnston.com> http://www.nathanieljohnston.com
%
%This work is licensed under the Creative Commons Attribution-Noncommercial-Share Alike 2.5 License. To view a copy of this license, visit http://creativecommons.org/licenses/by-nc-sa/2.5/ or send a letter to Creative Commons, 543 Howard Street, 5th Floor, San Francisco, California, 94105, USA.

\documentclass[letterpaper,11pt]{article}
\newlength{\outerbordwidth}
\pagestyle{empty}
\raggedbottom
\raggedright
\usepackage[svgnames]{xcolor}
\usepackage{framed}
\usepackage{tocloft}
\usepackage{etoolbox}
\usepackage{hyperref}
\usepackage[normalem]{ulem}
\usepackage{xparse}

\robustify\cftdotfill

\makeatletter
\newcommand*{\declarecommand}{%
    \@star@or@long\declare@command
}
\newcommand*{\declare@command}[1]{%
    \provide@command{#1}{}%
    % \let#1\@empty % would be more efficient, but without error checking
    \RenewDocumentCommand{#1}%
}
\makeatother

\hypersetup{
    colorlinks=false,
}

%-----------------------------------------------------------
%Edit these values as you see fit

\setlength{\outerbordwidth}{3pt}  % Width of border outside of title bars
\definecolor{shadecolor}{gray}{0.75}  % Outer background color of title bars (0 = black, 1 = white)
\definecolor{shadecolorB}{gray}{0.93}  % Inner background color of title bars


%-----------------------------------------------------------
%Margin setup

\setlength{\evensidemargin}{-0.25in}
\setlength{\headheight}{0in}
\setlength{\headsep}{0in}
\setlength{\oddsidemargin}{-0.25in}
\setlength{\paperheight}{11in}
\setlength{\paperwidth}{8.5in}
\setlength{\tabcolsep}{0in}
\setlength{\textheight}{9.5in}
\setlength{\textwidth}{7in}
\setlength{\topmargin}{-0.3in}
\setlength{\topskip}{0in}
\setlength{\voffset}{0.1in}


%-----------------------------------------------------------
%Custom commands
\newcommand{\resitem}[1]{\item #1 \vspace{-2pt}}
\newcommand{\resheading}[1]{\vspace{8pt}
  \parbox{\textwidth}{\setlength{\FrameSep}{\outerbordwidth}
    \begin{shaded}
\setlength{\fboxsep}{0pt}\framebox[\textwidth][l]{\setlength{\fboxsep}{4pt}\fcolorbox{shadecolorB}{shadecolorB}{\textbf{\sffamily{\mbox{~}\makebox[6.762in][l]{\large #1} \vphantom{p\^{E}}}}}}
    \end{shaded}
  }\vspace{-5pt}
}
\newcommand{\ressubheading}[4]{
\begin{tabular*}{6.5in}{l@{\cftdotfill{\cftsecdotsep}\extracolsep{\fill}}r}
		\textbf{#1} & #2 \\
		\textit{#3} & \textit{#4} \\
\end{tabular*}\vspace{-6pt}}
%-----------------------------------------------------------


\begin{document}

\begin{tabular*}{7in}{l@{\extracolsep{\fill}}r}
\textbf{\Large Bryan Bugyi} & \textbf{\today} \\
Software Engineer & bryanbugyi34@gmail.com \\
https://github.com/bbugyi200 & \\
\end{tabular*}
\\

%%%%%%%%%%%%%%%%%%%%%%%%%%%%%%
\resheading{Summary}
Student of Computer Science and Mathematics at Rutgers University, New Brunswick. Dedicated to writing clean, efficient code using best software development practices.
%%%%%%%%%%%%%%%%%%%%%%%%%%%%%%

%%%%%%%%%%%%%%%%%%%%%%%%%%%%%%
\resheading{Education}
%%%%%%%%%%%%%%%%%%%%%%%%%%%%%%
\begin{itemize}
\item
    \ressubheading{Rutgers University}{New Brunswick, NJ}{B.S. Computer Science, B.S. Mathematics (3.80 GPA)}{2017 - 2019}

\item
    \ressubheading{Rowan College at Burlington County}{Mount Laurel, NJ}{A.S. Computer Science (4.0 GPA)}{2015 - 2017}
\end{itemize}

%%%%%%%%%%%%%%%%%%%%%%%%%%%%%%
\resheading{Industry Experience}
%%%%%%%%%%%%%%%%%%%%%%%%%%%%%%
\begin{itemize}
\item
	\ressubheading{Comcast Cable}{Mount Laurel, NJ}{Tier III Technical Support}{2011 - 2016}
	\begin{itemize}
            \resitem{Monitored for large-scale product outages.}
            \resitem{Tracked and reported product outages using Jira ticketing system.}
            \resitem{Provided support and guidance to Tier II Technical Support representatives.}
	\end{itemize}
\end{itemize}

%%%%%%%%%%%%%%%%%%%%%%%%%%%%%%
\resheading{Open Source (GitHub) Projects}
%%%%%%%%%%%%%%%%%%%%%%%%%%%%%%

\begin{itemize}
\item \ressubheading{bbugyi200/cookie}{Shell}{A Template-based File Generator.}{2018 - Current}
    \begin{itemize}
        \item
            Well received by the developer community (over 150 stars on GitHub).
        \item
            Multiple outside contributions have been accepted (e.g. user submitted issues/bug reports and code contributions).
    \end{itemize}
\item \ressubheading{bbugyi200/funky}{Python, Shell}{Create per-directory local aliases in zsh.}{2017-Current}
    \begin{itemize}
        \item
            Well received by the developer community (over 300 stars on GitHub).
        \item
            Multiple outside contributions have been accepted (e.g. user submitted issues/bug reports and code contributions).
    \end{itemize}
\item \ressubheading{GermainZ/weechat-vimode}{Python}{A WeeChat script that adds vi-like modes, commands and keybindings.}{2017 - 2018}
    \begin{itemize}
        \item
            Made several non-trivial contributions to this project. My most significant contribution was a parser that I wrote to replicate vim's \texttt{:nmap} command ($\sim$500 lines of code).
    \end{itemize}
\item \ressubheading{HackRU/SlackRU}{Python, SQL}{Slackbot used for Rutgers HackRU Hackathon.}{2018}
    \begin{itemize}
        \item
            Used by hundreds of students at Rutger's 2018 HackRU Hackathon.
        \item
            Implemented using Slack's API and Python's Flask Web Framework (hosted on an AWS Elastic Beanstalk).
    \end{itemize}
\item \ressubheading{bbugyi200/WumpusWorld}{Python}{Autonomous Agent which navigates through Wumpus World.}{2016 - 2017}
    \begin{itemize}
        \item
            I presented a poster on this project at the 2017 Garden State Undergraduate Mathematics Conference (GSUMC).
    \end{itemize}
\end{itemize}

%%%%%%%%%%%%%%%%%%%%%%%%%%%%%%
\resheading{Activities}
%%%%%%%%%%%%%%%%%%%%%%%%%%%%%%
\begin{itemize}
    \item
        \ressubheading{bryanbugyi.com}{}{My Personal Tech/Programming Blog.}{2018-Current}
    \item 
        \ressubheading{HackRU Architect Team}{}{We Build Tools for HackRU (Rutgers Hackathon).}{2017-2018}
    \item
        \ressubheading{USACS Mentorship Program}{}{Mentor for CS Underclassmen.}{2017-2018}
    \item
        \ressubheading{Undergraduate Research in Mathematics}{}{Advised by Professor Jonathan Weisbrod.}{2017-2018}
\end{itemize}

%%%%%%%%%%%%%%%%%%%%%%%%%%%%%%
\resheading{Skills}
%%%%%%%%%%%%%%%%%%%%%%%%%%%%%%
\begin{itemize}
    \item {\bf Languages:}\\
        I have assigned each of the programming languages listed below a rank between 1-3 to indicate my level of proficiency with that language. A rank of 1 indicates that I have taken a college course that made use of the language and/or have read a book about the language. A rank of 2 indicates that I have also written over 5000 lines of code in the language. And a rank of 3 indicates that I have also written over 30000 lines of code in the language.\\\vspace{0.2cm}
        \textit{\uline{Programming Languages}}\\
        \begin{itemize}
            \item
                \textsc{Python}: 3
            \item
                \textsc{C/C++}: 2
            \item
                \textsc{Shell}: 2
            \item
                \textsc{Java}: 1
            \item
                \textsc{Haskell}: 1
            \item
                \textsc{Javascript}: 1
        \end{itemize}

        \textit{\uline{Markup / Domain-Specific Languages}}\\
        (The rank requirements are relaxed a bit in this section.)
        \begin{itemize}
            \item
                \LaTeX: 3
            \item
                \textsc{SQL}: 2
            \item
                \textsc{AWK}: 2
            \item
                \textsc{HTML/CSS}: 1
        \end{itemize}

    \item {\bf Technologies:} Flask, AWS, TravisCI, git, vim, tmux

    \item {\bf Linux:}

    \item {\bf Networking:}
\end{itemize}

\end{document}
