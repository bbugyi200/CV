% (c) 2002 Matthew Boedicker <mboedick@mboedick.org> (original author) http://mboedick.org
% (c) 2003-2007 David J. Grant <davidgrant-at-gmail.com> http://www.davidgrant.ca
% (c) 2008 Nathaniel Johnston <nathaniel@nathanieljohnston.com> http://www.nathanieljohnston.com
%
%This work is licensed under the Creative Commons Attribution-Noncommercial-Share Alike 2.5 License. To view a copy of this license, visit http://creativecommons.org/licenses/by-nc-sa/2.5/ or send a letter to Creative Commons, 543 Howard Street, 5th Floor, San Francisco, California, 94105, USA.

%%%%%%%%%%%%%%%%%%%%%%%%%%%%% Header Declarations
\documentclass[letterpaper,11pt]{article}
\usepackage{booktabs}
\usepackage{etoolbox}
\usepackage{framed}
\usepackage{tocloft}
\usepackage{hyperref}
\usepackage{multicol}
\usepackage{tabularx}
\usepackage[svgnames]{xcolor}

\input{/home/bryan/Sync/lib/latex/gutils.tex}

%%%%%%%%%%%%%%%%%%%%%%%%%%%%% Style Settings
\newlength{\outerbordwidth}
\pagestyle{empty}
\raggedbottom
\raggedright

\robustify\cftdotfill

\hypersetup{
    colorlinks=false,
}

\setlength{\outerbordwidth}{3pt}  % Width of border outside of title bars
\definecolor{shadecolor}{gray}{0.75}  % Outer background color of title bars (0 = black, 1 = white)
\definecolor{shadecolorB}{gray}{0.93}  % Inner background color of title bars


% ----- Margin
\setlength{\evensidemargin}{-0.25in}
\setlength{\headheight}{0in}
\setlength{\headsep}{0in}
\setlength{\oddsidemargin}{-0.25in}
\setlength{\paperheight}{11in}
\setlength{\paperwidth}{8.5in}
\setlength{\tabcolsep}{0in}
\setlength{\textheight}{9.5in}
\setlength{\textwidth}{7in}
\setlength{\topmargin}{-0.3in}
\setlength{\topskip}{0in}
\setlength{\voffset}{0.1in}


%%%%%%%%%%%%%%%%%%%%%%%%%%%%% Custom Commands
\newcommand{\resitem}[1]{\item #1 \vspace{-2pt}}
\newcommand{\resheading}[1]{\vspace{8pt}
  \parbox{\textwidth}{\setlength{\FrameSep}{\outerbordwidth}
    \begin{shaded}
\setlength{\fboxsep}{0pt}\framebox[\textwidth][l]{\setlength{\fboxsep}{4pt}\fcolorbox{shadecolorB}{shadecolorB}{\textbf{\sffamily{\mbox{~}\makebox[6.762in][l]{\large #1} \vphantom{p\^{E}}}}}}
    \end{shaded}
  }\vspace{-5pt}
}
\newcommand{\ressubheading}[4]{
\begin{tabular*}{6.5in}{l@{\cftdotfill{\cftsecdotsep}\extracolsep{\fill}}r}
		\textbf{#1} & #2 \\
		\textit{#3} & \textit{#4} \\
\end{tabular*}\vspace{-6pt}}


%%%%%%%%%%%%%%%%%%%%%%%%%%%%% Start of CV
\begin{document}

\begin{tabular*}{7in}{l@{\extracolsep{\fill}}r}
\textbf{\Large Bryan Bugyi} & \textbf{\today} \\
Software Engineer & bryanbugyi34@gmail.com \\
    https://linkedin.com/in/bryan-bugyi & (609)500-7081\\
\end{tabular*}
\\

%%%%%%%%%%%%%%%%%%%%%%%%%%%%%%%%%%%%%%%%%%%%%%%%%%%%%%%%%%%
%  SUMMARY                                                %
%%%%%%%%%%%%%%%%%%%%%%%%%%%%%%%%%%%%%%%%%%%%%%%%%%%%%%%%%%%
\resheading{Summary}
I am a Software Engineer with a very ``systems''-oriented skill set. In particular, I have the following areas of expertise / specialization: \hspace{1pt} \textbf{Python} programming, \textbf{Linux}, and \textbf{Networking} (TCP/IP)

%%%%%%%%%%%%%%%%%%%%%%%%%%%%%%%%%%%%%%%%%%%%%%%%%%%%%%%%%%%
%  EDUCATION                                              %
%%%%%%%%%%%%%%%%%%%%%%%%%%%%%%%%%%%%%%%%%%%%%%%%%%%%%%%%%%%
\resheading{Education}
\begin{itemize}
\item
    \ressubheading{Rutgers University}{New Brunswick, NJ}{B.S. Computer Science (minor in Mathematics)}{2015 - 2019}
\end{itemize}

%%%%%%%%%%%%%%%%%%%%%%%%%%%%%%%%%%%%%%%%%%%%%%%%%%%%%%%%%%%
%  INDUSTRY EXPERIENCE                                    %
%%%%%%%%%%%%%%%%%%%%%%%%%%%%%%%%%%%%%%%%%%%%%%%%%%%%%%%%%%%
\resheading{Industry Experience}
\begin{itemize}
    \item
    \ressubheading{Bloomberg L.P.}{Manhattan, NY}{Site Reliability Engineer (SRE) / Senior Software Engineer}{January 2021 - Current} \vspace{8pt}

        SREs at Bloomberg use both software and systems engineering to build and reliably run their production systems. I am a founding member of the Compliance SRE (CSRE) team. A few notable achievements associated with my work at Bloomberg are listed below:
	\begin{itemize}
            \resitem{Lead Software Engineer for most software projects (primarily Python) on the CSRE team.}
            \resitem{Maintainer of Bloomberg's community cookiecutter for Python projects.}
            \resitem{Founder of new ``ChangeLog Driven Release (CLDR)'' methodology (alpha testing across Compliance department).}
	\end{itemize}

    \item
	\ressubheading{Edgestream Partners, L.P.}{Princeton, NJ}{Software Engineer}{May 2019 - January 2021} \vspace{8pt}

        As a Software Engineer on the Production team at Edgestream, my work centered around the improvement / maintenance of Edgestream's production trading system. This system is Linux-based (Red Hat) and is written primarily in Python. A few notable achievements associated with my work at Edgestream are listed below: \vspace{-4pt}
	\begin{itemize}
            \resitem{Integrated pylint into a codebase with well over a million lines of existing code. As the lead on this project, I was responsible for building several non-trivial tools to aid in making this integration a success.}
            \resitem{Made several large-scale improvements to the production department's in-house testing framework/runner.}
            \resitem{Lead developer of Edgestream's investor-facing web portal, which was built using Python's Django web framework.}
	\end{itemize}

    \item
    \ressubheading{Comcast}{Mount Laurel, NJ}{Tier III Technical Support}{2011 - 2016} \vspace{8pt}

        Tier III Technical Support at Comcast was a liaison group that worked to resolve chronic customer issues as well as outages which were impacting multiple customers. This involved troubleshooting Comcast's network to resolve the issue directly or, failing at that, escalating to the appropriate Engineering group and working with them to diagnose and resolve the issue cooperatively. Some of the other responsibilities associated with this role are listed below: \vspace{-4pt}
    \begin{itemize}
            \resitem{Monitored for large-scale network outages by aggregating and analyzing customer issue data from our ticketing system.} 
            \resitem{Used Python scripting to automate frequent department tasks (e.g. parsing spreadsheets, opening/closing tickets, information retrieval).}
    \end{itemize}
\end{itemize}

%%%%%%%%%%%%%%%%%%%%%%%%%%%%%%%%%%%%%%%%%%%%%%%%%%%%%%%%%%%
%  SKILLS                                                 %
%%%%%%%%%%%%%%%%%%%%%%%%%%%%%%%%%%%%%%%%%%%%%%%%%%%%%%%%%%%
\resheading{Skills}
\begin{itemize}
    \declarecommand{\sub}{m}{\item {\bf #1:}}
    \declarecommand{\ssub}{m}{\textit{\uline{#1}}\\}

    \sub{Programming Languages}\\
        To indicate my level of proficiency with a language, I have used ``lines-of-code ($LoC$) written'' as my key metric, though I admit this provides at best an incomplete picture:

        \declarecommand{\lang}{m m}{\textsc{#1} $\geq$ #2 $LoC$}

        \declarecommand{\A}{}{$1,000$}
        \declarecommand{\B}{}{$5,000$}
        \declarecommand{\C}{}{$20,000$} 
        \declarecommand{\D}{}{$50,000$}
        \declarecommand{\E}{}{$100,000$}

        \vspace{0.08cm}
        \begin{tabularx}{0.9\textwidth} { >{\raggedright\arraybackslash}X  >{\raggedright\arraybackslash}X  >{\raggedright\arraybackslash}X }
            \lang{Python}{\E} &
            \lang{Bash}{\C} &
            \lang{C/C++}{\C} \\
            \lang{Perl}{\B} &
            \lang{Rust}{\B} &
            \lang{Vimscript}{\B} \\
            \lang{Haskell}{\A} &
            \lang{Java}{\A} &
            \lang{Javascript}{\A} \\
        \end{tabularx}

    \sub{Frameworks} Django, FastAPI, Flask, Pluggy, ZeroMQ

    \sub{Technologies} Docker, Git, \LaTeX, Linux, PostgreSQL, Redis, Vim
\end{itemize}

%%%%%%%%%%%%%%%%%%%%%%%%%%%%%%%%%%%%%%%%%%%%%%%%%%%%%%%%%%%
%  OPEN SOURCE PROJECTS                                   %
%%%%%%%%%%%%%%%%%%%%%%%%%%%%%%%%%%%%%%%%%%%%%%%%%%%%%%%%%%%
\resheading{Open Source (GitHub) Projects}

\begin{itemize}
\item \ressubheading{python-boltons/*}{Python}{Python libraries made available as ``boltons''; We think they should be ``builtins''.}{2021--present}
    \begin{itemize}
        \item
            Founder and lead developer of this project / series of projects. Example bolton libraries:\vspace{-0.57cm}
            \begin{multicols}{3}
                \begin{itemize}
                    \item cc-python (cookiecutter)
                    \item clack (config CLI lib)
                    \item eris (error-handling lib)
                    \item hush (plugin secrets lib)
                    \item logrus (logging lib)
                \end{itemize}
            \end{multicols}\vspace{-0.45cm}
    \end{itemize}
\item \ressubheading{psf/black}{Python}{The uncompromising Python code formatter.}{2019--2021}
    \begin{itemize}
        \item
            Improved the way black handles strings (https://github.com/psf/black/pull/1132).
        \item
            Non-trivial contribution consisting of $\sim$3,500 lines of code additions/modifications.
        \item
            Led to the closing of five unrelated GitHub issues (opened by five different developers at different times).
    \end{itemize}
\item \ressubheading{bbugyi200/funky}{Python, Shell}{Makes shell functions easier to define, more flexible, and more interactive.}{2017--2021}
    \begin{itemize}
        \item
            Well received by the developer community (over 500 stars on GitHub).
        \item
            Multiple outside contributions have been accepted (e.g. user submitted issues/bug reports and code contributions).
    \end{itemize}
\item \ressubheading{bbugyi200/cookie}{Shell}{A Template-based File Generator.}{2018--2019}
    \begin{itemize}
        \item
            Well received by the developer community (over 250 stars on GitHub).
        \item
            Multiple outside contributions have been accepted (e.g. user submitted issues/bug reports and code contributions).
    \end{itemize}
\item \ressubheading{bbugyi200/*}{Python, Rust, Shell}{Miscellaneous projects that I've open-sourced on GitHub.}{2017--present}
    \begin{itemize}[noitemsep]
        \item greatday (GTD todo manager) | Python
        \item cldr (ChangeLog Driven Releases) | Python
        \item shv (Shell History Viewer) | Rust, Shell
    \end{itemize}
\end{itemize}

\end{document}
