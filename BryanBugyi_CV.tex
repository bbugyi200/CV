% (c) 2002 Matthew Boedicker <mboedick@mboedick.org> (original author) http://mboedick.org
% (c) 2003-2007 David J. Grant <davidgrant-at-gmail.com> http://www.davidgrant.ca
% (c) 2008 Nathaniel Johnston <nathaniel@nathanieljohnston.com> http://www.nathanieljohnston.com
%
%This work is licensed under the Creative Commons Attribution-Noncommercial-Share Alike 2.5 License. To view a copy of this license, visit http://creativecommons.org/licenses/by-nc-sa/2.5/ or send a letter to Creative Commons, 543 Howard Street, 5th Floor, San Francisco, California, 94105, USA.

%%%%%%%%%%%%%%%%%%%%%%%%%%%%% Header Declarations
\documentclass[letterpaper,11pt]{article}
\usepackage{etoolbox}
\usepackage{framed}
\usepackage{tocloft}
\usepackage{hyperref}
\usepackage[svgnames]{xcolor}

\input{/home/bryan/Sync/lib/latex/gutils.tex}

%%%%%%%%%%%%%%%%%%%%%%%%%%%%% Style Settings
\newlength{\outerbordwidth}
\pagestyle{empty}
\raggedbottom
\raggedright

\robustify\cftdotfill

\hypersetup{
    colorlinks=false,
}

\setlength{\outerbordwidth}{3pt}  % Width of border outside of title bars
\definecolor{shadecolor}{gray}{0.75}  % Outer background color of title bars (0 = black, 1 = white)
\definecolor{shadecolorB}{gray}{0.93}  % Inner background color of title bars


% ----- Margin
\setlength{\evensidemargin}{-0.25in}
\setlength{\headheight}{0in}
\setlength{\headsep}{0in}
\setlength{\oddsidemargin}{-0.25in}
\setlength{\paperheight}{11in}
\setlength{\paperwidth}{8.5in}
\setlength{\tabcolsep}{0in}
\setlength{\textheight}{9.5in}
\setlength{\textwidth}{7in}
\setlength{\topmargin}{-0.3in}
\setlength{\topskip}{0in}
\setlength{\voffset}{0.1in}


%%%%%%%%%%%%%%%%%%%%%%%%%%%%% Custom Commands
\newcommand{\resitem}[1]{\item #1 \vspace{-2pt}}
\newcommand{\resheading}[1]{\vspace{8pt}
  \parbox{\textwidth}{\setlength{\FrameSep}{\outerbordwidth}
    \begin{shaded}
\setlength{\fboxsep}{0pt}\framebox[\textwidth][l]{\setlength{\fboxsep}{4pt}\fcolorbox{shadecolorB}{shadecolorB}{\textbf{\sffamily{\mbox{~}\makebox[6.762in][l]{\large #1} \vphantom{p\^{E}}}}}}
    \end{shaded}
  }\vspace{-5pt}
}
\newcommand{\ressubheading}[4]{
\begin{tabular*}{6.5in}{l@{\cftdotfill{\cftsecdotsep}\extracolsep{\fill}}r}
		\textbf{#1} & #2 \\
		\textit{#3} & \textit{#4} \\
\end{tabular*}\vspace{-6pt}}


%%%%%%%%%%%%%%%%%%%%%%%%%%%%% Start of CV
\begin{document}

\begin{tabular*}{7in}{l@{\extracolsep{\fill}}r}
\textbf{\Large Bryan Bugyi} & \textbf{\today} \\
Software Engineer & bryanbugyi34@gmail.com \\
    https://github.com/bbugyi200 & (609)500-7081\\
\end{tabular*}
\\

%%%%%%%%%%%%%%%%%%%%%%%%%%%%%%%%%%%%%%%%%%%%%%%%%%%%%%%%%%%
%  SUMMARY                                                %
%%%%%%%%%%%%%%%%%%%%%%%%%%%%%%%%%%%%%%%%%%%%%%%%%%%%%%%%%%%
\resheading{Summary}
Software Engineer at Edgestream Partners, L.P. in Princeton, NJ where I was hired to maintain and improve the company's large pre-existing codebase.

\vspace{0.3cm}
I am most effective when working with Python or C/C++ in a \textbf{Linux} environment.

%%%%%%%%%%%%%%%%%%%%%%%%%%%%%%%%%%%%%%%%%%%%%%%%%%%%%%%%%%%
%  EDUCATION                                              %
%%%%%%%%%%%%%%%%%%%%%%%%%%%%%%%%%%%%%%%%%%%%%%%%%%%%%%%%%%%
\resheading{Education}
\begin{itemize}
\item
    \ressubheading{Rutgers University}{New Brunswick, NJ}{B.S. Computer Science w/ minor in Mathematics}{2015 - 2019}
\end{itemize}

%%%%%%%%%%%%%%%%%%%%%%%%%%%%%%%%%%%%%%%%%%%%%%%%%%%%%%%%%%%
%  INDUSTRY EXPERIENCE                                    %
%%%%%%%%%%%%%%%%%%%%%%%%%%%%%%%%%%%%%%%%%%%%%%%%%%%%%%%%%%%
\resheading{Industry Experience}
\begin{itemize}
\item
	\ressubheading{Edgestream Partners, L.P.}{Princeton, NJ}{Software Engineer}{May 2019 - Current}
	\begin{itemize}
            \resitem{Integrated pylint into a codebase with well over a million lines of existing code. As the lead on this project, I was responsible for building several non-trivial tools to aid in making this integration a success.}
            \resitem{Made several large-scale improvements to the production department's in-house testing framework/runner.}
            \resitem{Lead developer of Edgestream's investor-facing web portal, which was built using Python's Django web framework.}
	\end{itemize}
    \item
        \ressubheading{Comcast}{Mount Laurel, NJ}{Tier III Technical Support}{2011 - 2016}
        \begin{itemize}
                \resitem{Monitored for large-scale network outages.}
                \resitem{Tracked and reported network outages using ticketing system.}
                \resitem{Developed Python scripts to automate frequent department tasks (e.g. parsing spreadsheets, opening/closing tickets, information retrieval).} 
        \end{itemize}
\end{itemize}

%%%%%%%%%%%%%%%%%%%%%%%%%%%%%%%%%%%%%%%%%%%%%%%%%%%%%%%%%%%
%  SKILLS                                                 %
%%%%%%%%%%%%%%%%%%%%%%%%%%%%%%%%%%%%%%%%%%%%%%%%%%%%%%%%%%%
\resheading{Skills}
\begin{itemize}
    \declarecommand{\sub}{m}{\item {\bf #1:}}
    \declarecommand{\ssub}{m}{\textit{\uline{#1}}\\}

    \sub{Languages}\\
        I have assigned each of the languages listed below a rank between 1-4 to indicate my level of proficiency with that language. A rank of 1 indicates that I have taken a college course that made use of the language and/or have read a book about the language. A rank of 2 indicates that I have also written over 5,000 lines of code in the language. A rank of 3 indicates that I have also written over 20,000 lines of code in the language. And a rank of 4 indicates that I have also written over 50,000 lines of code in the language.\\\vspace{0.2cm}

        \declarecommand{\lang}{m m}{\item \textsc{#1}: #2}

        \ssub{Programming Languages}
        \begin{itemize}
            \lang{Python}{4}
            \lang{C/C++}{3}
            \lang{Shell}{3}
            \lang{Perl}{2}
            \lang{Rust}{2}
            \lang{Haskell}{1}
            \lang{Java}{1}
            \lang{Javascript}{1}
        \end{itemize}

    \sub{Frameworks} Django, Flask, Pandas, Twisted
    \sub{Technologies} AWS, Docker, Git, PostgreSQL, RedHat, Vim
    \sub{Specializations} Linux, Systems Programming, Networking (TCP/IP), Language Parsers
\end{itemize}

%%%%%%%%%%%%%%%%%%%%%%%%%%%%%%%%%%%%%%%%%%%%%%%%%%%%%%%%%%%
%  OPEN SOURCE PROJECTS                                   %
%%%%%%%%%%%%%%%%%%%%%%%%%%%%%%%%%%%%%%%%%%%%%%%%%%%%%%%%%%%
\resheading{Open Source (GitHub) Projects}

\begin{itemize}
\item \ressubheading{psf/black}{Python}{The uncompromising Python code formatter.}{2019 - 2020}
    \begin{itemize}
        \item
            Improved the way black handles strings (https://github.com/psf/black/pull/1132).
        \item
            Non-trivial contribution consisting of $\sim$3,500 lines of code additions/modifications.
        \item
            Led to the closing of five unrelated GitHub issues (opened by five different developers at different times).
    \end{itemize}
\item \ressubheading{bbugyi200/cookie}{Shell}{A Template-based File Generator.}{2018 - 2019}
    \begin{itemize}
        \item
            Well received by the developer community (over 200 stars on GitHub).
        \item
            Multiple outside contributions have been accepted (e.g. user submitted issues/bug reports and code contributions).
    \end{itemize}
\item \ressubheading{bbugyi200/funky}{Python, Shell}{Makes shell functions easier to define, more flexible, and more interactive.}{2017 - 2019}
    \begin{itemize}
        \item
            Well received by the developer community (over 400 stars on GitHub).
        \item
            Multiple outside contributions have been accepted (e.g. user submitted issues/bug reports and code contributions).
    \end{itemize}
\item \ressubheading{GermainZ/weechat-vimode}{Python}{A WeeChat script that adds vi-like modes, commands and keybindings.}{2017 - 2018}
    \begin{itemize}
        \item
            Made several non-trivial contributions to this project. My most significant contribution was a parser that I wrote which was used to replicate vim's \texttt{:nmap} command ($\sim$500 lines of code).
    \end{itemize}
\end{itemize}

\end{document}
